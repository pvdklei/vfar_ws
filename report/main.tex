% main.tex - Main document file for Lab 1 Assignment Report
\documentclass[11pt,a4paper]{article}

% ==================== PACKAGES ====================
\usepackage[utf8]{inputenc}
\usepackage[english]{babel}
\usepackage{geometry}
\usepackage{graphicx}
\usepackage{float}
\usepackage{amsmath}
\usepackage{amssymb}
\usepackage{hyperref}
\usepackage{caption}
\usepackage{subcaption}
\usepackage{booktabs}
\usepackage{algorithm}
\usepackage{algpseudocode}
\usepackage{xcolor}
\usepackage{listings}
\usepackage{enumitem}

% ==================== PAGE LAYOUT ====================
\geometry{
    a4paper,
    left=25mm,
    right=25mm,
    top=30mm,
    bottom=30mm
}

% ==================== HYPERREF SETUP ====================
\hypersetup{
    colorlinks=true,
    linkcolor=blue,
    filecolor=magenta,
    urlcolor=cyan,
    citecolor=blue,
    pdftitle={Lab 1: Line Detection and Following},
    pdfauthor={Student IDs},
}

% ==================== CODE LISTING SETUP ====================
\lstset{
    basicstyle=\ttfamily\small,
    breaklines=true,
    frame=single,
    numbers=left,
    numberstyle=\tiny\color{gray},
    commentstyle=\color{green!60!black},
    keywordstyle=\color{blue},
    stringstyle=\color{red}
}

% ==================== TITLE INFORMATION ====================
\title{
    \textbf{Visual Feedback for Autonomous Robots}\\
    \large Lab 1: Line Detection and Following by UGV Rover
}
\author{
    Student Name 1 (ID1) \and
    Student Name 2 (ID2) \and
    Student Name 3 (ID3) \and
    Student Name 4 (ID4)
}
\date{\today}

% ==================== DOCUMENT BEGINS ====================
\begin{document}

\maketitle

\begin{abstract}
% PLACEHOLDER: Write a concise abstract (150-200 words) that summarizes:
% - The objective of the assignment (line detection and following with UGV Rover)
% - Main approaches used (Canny edge detection, alternative methods, control algorithms)
% - Key findings (which algorithms worked best, challenges encountered)
% - Main conclusion (success of line following implementation)
This abstract should briefly describe the entire lab assignment, highlighting the main objectives, methodologies, and outcomes of both Session 1 (line detection) and Session 2 (line following).
\end{abstract}

\tableofcontents
\newpage

% ==================== MAIN SECTIONS ====================
% introduction.tex - Introduction section
\section{Introduction}
\label{sec:introduction}

% PLACEHOLDER: Opening paragraph (2-3 sentences)
% - Introduce the general context: autonomous robots and their need for visual feedback
% - Mention the importance of line following in robotics applications
% - Example: "Autonomous mobile robots rely heavily on visual feedback to navigate their environment.
%   Line following is a fundamental task in robotics, with applications ranging from warehouse automation
%   to autonomous vehicles maintaining lane positions."

\subsection{Motivation and Background}
\label{subsec:motivation}

% PLACEHOLDER: Motivation paragraph (3-4 sentences)
% - Explain why line detection and following is important
% - Discuss real-world applications (autonomous vehicles, warehouse robots, etc.)
% - Mention the specific context of this lab (UGV Rover following ceiling lines)
% - Example: "In industrial environments, autonomous guided vehicles (AGVs) use line following to
%   transport materials efficiently. This assignment explores line following in a unique scenario:
%   following ceiling-mounted light lines rather than ground markers."

\subsection{Assignment Objectives}
\label{subsec:objectives}

% PLACEHOLDER: List the main objectives of the assignment
% This should be a clear enumeration of what the lab aims to achieve:
The main objectives of this assignment are:

\begin{enumerate}
    \item \textbf{Line Detection (Session 1):}
    % PLACEHOLDER: Describe the line detection objectives
    % - Design and implement a line detection pipeline
    % - Evaluate Canny edge detector performance
    % - Test on provided rosbag data
    % - Explore and compare alternative line detection algorithms

    \item \textbf{Line Following (Session 2):}
    % PLACEHOLDER: Describe the line following objectives
    % - Implement a complete line following system
    % - Develop tracking algorithm to follow specific ceiling line
    % - Evaluate impact of camera calibration
    % - Address control challenges (e.g., oscillation, stability)

    \item \textbf{System Integration:}
    % PLACEHOLDER: Describe integration objectives
    % - Integrate computer vision algorithms with ROS2
    % - Implement real-time control based on visual feedback
    % - Test and validate on physical UGV Rover platform
\end{enumerate}

\subsection{Technical Approach}
\label{subsec:approach}

% PLACEHOLDER: Brief overview of technical approach (3-5 sentences)
% - Mention the UGV Rover platform and its capabilities
% - Describe the ROS2 framework used
% - Outline the general pipeline: image capture → processing → line detection → control → actuation
% - Mention key tools: OpenCV for image processing, Python for implementation
% - Example: "This assignment uses a UGV Rover equipped with a pan-tilt camera to capture images of
%   ceiling lines. The Robot Operating System 2 (ROS2) framework provides the infrastructure for
%   real-time image processing and robot control."

\subsection{Report Organization}
\label{subsec:organization}

% PLACEHOLDER: Outline the structure of the report
% This should match the actual structure of your document
The remainder of this report is organized as follows:
\begin{itemize}
    \item Section \ref{sec:session1} describes the line detection pipeline design and implementation, including comparison of different edge detection algorithms.
    \item Section \ref{sec:session2} presents the line following system, including tracking algorithms, camera calibration effects, and control improvements.
    \item Section \ref{sec:conclusion} summarizes the key findings, discusses limitations, and suggests future improvements.
\end{itemize}


% session1.tex - Session 1: Line Detection
\section{Session 1: Line Detection by UGV Rover}
\label{sec:session1}

% PLACEHOLDER: Brief introduction to Session 1 (2-3 sentences)
% - Explain that this section covers the line detection phase
% - Mention that the goal is to detect ceiling lines from camera images
% - Preview the subsections: pipeline design, Canny implementation, full run testing, alternative algorithms

\subsection{Line Detection Pipeline Design}
\label{subsec:pipeline_design}

% PLACEHOLDER: Describe your pipeline design (10 points)
% This subsection should include:
% 1. A clear description of the input (raw camera images from UGV Rover)
% 2. Each processing step in your pipeline
% 3. The output (detected lines)
% 4. Justification for each step

\subsubsection{Pipeline Overview}

% PLACEHOLDER: Write 2-3 sentences introducing your pipeline
% - Mention the main stages (e.g., preprocessing, edge detection, line extraction, filtering)
% - Explain the overall flow from input to output

\begin{figure}[H]
    \centering
    % PLACEHOLDER: Insert a flowchart or block diagram showing your pipeline
    % The diagram should show: Camera Image → Preprocessing → Edge Detection → Line Extraction → Filtering → Output
    % You can create this using TikZ, include an external image, or use the algorithm environment
    \fbox{\parbox{0.8\textwidth}{\centering
        \textbf{[FIGURE: Pipeline Flowchart]}\\[1em]
        Insert a flowchart showing:\\
        Raw Image $\rightarrow$ Grayscale Conversion $\rightarrow$ Noise Reduction $\rightarrow$\\
        Edge Detection $\rightarrow$ Line Extraction $\rightarrow$ Brightness Filtering $\rightarrow$ Output
    }}
    \caption{Line detection pipeline architecture. PLACEHOLDER: This figure should illustrate the complete pipeline from raw camera input to final line detection output.}
    \label{fig:pipeline}
\end{figure}

\subsubsection{Pipeline Steps}

% PLACEHOLDER: Describe each step in detail
% For each step, explain WHAT it does and WHY it's necessary

\paragraph{Step 1: Image Acquisition}
% PLACEHOLDER: Describe how images are captured from the UGV Rover camera
% - Mention the ROS2 topic subscribed to
% - Image format and resolution
% - Frame rate considerations

\paragraph{Step 2: Preprocessing}
% PLACEHOLDER: Describe preprocessing steps
% - Grayscale conversion (why: reduce computational complexity)
% - Gaussian blur or median filtering (why: reduce noise for better edge detection)
% - Contrast enhancement if needed (why: make ceiling lines more prominent)

\paragraph{Step 3: Edge Detection}
% PLACEHOLDER: Describe the edge detection stage
% - Which algorithm is used (Canny for initial implementation)
% - Parameter settings
% - Why: edges correspond to line boundaries

\paragraph{Step 4: Line Extraction}
% PLACEHOLDER: Describe how lines are extracted from edge maps
% - Hough Transform or contour detection
% - Why: convert pixel-level edges into line representations

\paragraph{Step 5: Line Filtering}
% PLACEHOLDER: Describe how ceiling lines are distinguished from other lines
% - Brightness-based filtering (ceiling lines are brighter)
% - Position filtering (ceiling lines appear in upper portion of image)
% - Orientation filtering (ceiling lines are mostly horizontal/vertical)
% - Why: focus only on target ceiling lines

\subsection{Implementation with Canny Edge Detector}
\label{subsec:canny_implementation}

% PLACEHOLDER: Describe Canny implementation (10 points)
% This subsection should include:
% 1. Brief description of how Canny edge detector works
% 2. Implementation details (OpenCV functions used, parameters chosen)
% 3. Five different detection results showing various scenarios

\subsubsection{Canny Edge Detector Overview}

% PLACEHOLDER: Write 3-4 sentences about Canny edge detector
% - How it works (gradient calculation, non-maximum suppression, hysteresis thresholding)
% - Why it's commonly used (optimal edge detection, good localization)
% - Parameters used: lower and upper thresholds

\subsubsection{Implementation Details}

% PLACEHOLDER: Describe implementation
% - OpenCV functions: cv2.Canny(), cv2.GaussianBlur()
% - Parameter values chosen and rationale
% - Any preprocessing applied before Canny

\begin{lstlisting}[language=Python, caption={Canny edge detection implementation (simplified). PLACEHOLDER: Replace with your actual implementation code.}]
import cv2
import numpy as np

def detect_lines_canny(image):
    # PLACEHOLDER: Insert your actual implementation code
    # This should include:
    # - Grayscale conversion
    # - Gaussian blur
    # - Canny edge detection
    # - Line extraction (Hough transform)
    # - Filtering for ceiling lines

    gray = cv2.cvtColor(image, cv2.COLOR_BGR2GRAY)
    blurred = cv2.GaussianBlur(gray, (5, 5), 0)
    edges = cv2.Canny(blurred, threshold1=50, threshold2=150)

    # Further processing...
    return detected_lines
\end{lstlisting}

\subsubsection{Detection Results}

% PLACEHOLDER: Show five different detection results
% For each result, include:
% - The input image
% - The processed output showing detected lines
% - Brief caption explaining the scenario (e.g., "Well-lit corridor", "Multiple lines visible", etc.)

\begin{figure}[H]
    \centering
    \begin{subfigure}[b]{0.45\textwidth}
        \centering
        % PLACEHOLDER: Insert actual image
        \fbox{\parbox{0.9\textwidth}{\centering [Input Image 1]}}
        \caption{Input image}
    \end{subfigure}
    \hfill
    \begin{subfigure}[b]{0.45\textwidth}
        \centering
        % PLACEHOLDER: Insert actual result
        \fbox{\parbox{0.9\textwidth}{\centering [Result 1]}}
        \caption{Detected edges}
    \end{subfigure}
    \caption{Result 1: PLACEHOLDER - Describe the scenario (e.g., "Ideal conditions with clear ceiling line")}
    \label{fig:canny_result1}
\end{figure}

\begin{figure}[H]
    \centering
    \begin{subfigure}[b]{0.45\textwidth}
        \centering
        \fbox{\parbox{0.9\textwidth}{\centering [Input Image 2]}}
        \caption{Input image}
    \end{subfigure}
    \hfill
    \begin{subfigure}[b]{0.45\textwidth}
        \centering
        \fbox{\parbox{0.9\textwidth}{\centering [Result 2]}}
        \caption{Detected edges}
    \end{subfigure}
    \caption{Result 2: PLACEHOLDER - Describe the scenario (e.g., "Multiple lines visible, including ground lines")}
    \label{fig:canny_result2}
\end{figure}

% PLACEHOLDER: Add three more similar figure sets for Results 3, 4, and 5
% Each should show different scenarios:
% - Different lighting conditions
% - Different numbers of lines visible
% - Different challenging situations (occlusions, shadows, etc.)

\subsection{Testing on Full Run}
\label{subsec:full_run}

% PLACEHOLDER: Describe testing on rosbag (10 points)
% This subsection should include:
% 1. How the algorithm performs on the provided rosbag
% 2. Three success cases and/or three failure cases with images
% 3. Analysis of why successes worked and failures occurred

\subsubsection{Overall Performance}

% PLACEHOLDER: Describe overall performance (4-5 sentences)
% - How well the algorithm handles the complete rosbag sequence
% - Success rate (qualitative or quantitative)
% - Common patterns observed
% - Main challenges encountered

\subsubsection{Success Cases}

% PLACEHOLDER: Show three impressive success cases
% For each, include image and explanation of why it succeeded

\begin{figure}[H]
    \centering
    % PLACEHOLDER: Insert success case image
    \fbox{\parbox{0.7\textwidth}{\centering [Success Case 1]}}
    \caption{Success Case 1: PLACEHOLDER - Explain why this detection worked well (e.g., "Clear ceiling line with high contrast and minimal interference from other objects")}
    \label{fig:success1}
\end{figure}

% PLACEHOLDER: Add two more success cases (figures and captions)

\subsubsection{Failure Cases}

% PLACEHOLDER: Show three failure cases
% For each, include image and explanation of why it failed

\begin{figure}[H]
    \centering
    % PLACEHOLDER: Insert failure case image
    \fbox{\parbox{0.7\textwidth}{\centering [Failure Case 1]}}
    \caption{Failure Case 1: PLACEHOLDER - Explain why this detection failed (e.g., "Window edges and ground lines interfere with ceiling line detection")}
    \label{fig:failure1}
\end{figure}

% PLACEHOLDER: Add two more failure cases (figures and captions)

\subsubsection{Analysis}

% PLACEHOLDER: Analyze the successes and failures (4-5 sentences)
% - What conditions lead to successful detection?
% - What conditions cause failures?
% - What are the limitations of Canny-based approach?
% - What improvements might be needed?

\subsection{Alternative Line Detection Algorithm}
\label{subsec:alternative_algorithm}

% PLACEHOLDER: Describe your alternative algorithm (10 points)
% This subsection should include:
% 1. Design of alternative algorithm (not Canny)
% 2. Comparison with Canny results
% 3. Analysis of why it works better or worse

\subsubsection{Algorithm Design}

% PLACEHOLDER: Describe your alternative approach (5-6 sentences)
% Options could include:
% - Hough Transform based approach
% - Sobel or Prewitt edge detection
% - Machine learning based line detection
% - Color-based segmentation followed by line fitting
% - Probabilistic Hough Transform
% Explain the key differences from Canny approach

\subsubsection{Implementation}

% PLACEHOLDER: Describe implementation details
% - Specific algorithms/functions used
% - Parameter tuning
% - Any novel modifications or combinations

\begin{algorithm}[H]
\caption{Alternative Line Detection Algorithm. PLACEHOLDER: Replace with your actual algorithm}
\label{alg:alternative}
\begin{algorithmic}[1]
\State \textbf{Input:} Raw camera image $I$
\State \textbf{Output:} Detected ceiling lines $L$
\State
\State \texttt{// PLACEHOLDER: Insert your algorithm steps}
\State $I_{gray} \gets \text{ConvertToGrayscale}(I)$
\State $I_{processed} \gets \text{ApplyPreprocessing}(I_{gray})$
\State $features \gets \text{ExtractFeatures}(I_{processed})$
\State $lines \gets \text{DetectLines}(features)$
\State $L \gets \text{FilterCeilingLines}(lines)$
\State \Return $L$
\end{algorithmic}
\end{algorithm}

\subsubsection{Comparison with Canny Detector}

% PLACEHOLDER: Compare your alternative with Canny
% Show side-by-side results on same images

\begin{figure}[H]
    \centering
    \begin{subfigure}[b]{0.45\textwidth}
        \centering
        \fbox{\parbox{0.9\textwidth}{\centering [Canny Result]}}
        \caption{Canny edge detector}
    \end{subfigure}
    \hfill
    \begin{subfigure}[b]{0.45\textwidth}
        \centering
        \fbox{\parbox{0.9\textwidth}{\centering [Alternative Result]}}
        \caption{Alternative algorithm}
    \end{subfigure}
    \caption{Comparison 1: PLACEHOLDER - Describe the comparison (e.g., "Alternative method better suppresses ground lines")}
    \label{fig:comparison1}
\end{figure}

% PLACEHOLDER: Add 2-3 more comparison figures

\subsubsection{Analysis and Discussion}

% PLACEHOLDER: Analyze why your alternative works better or worse (5-7 sentences)
% - Quantitative comparison if possible (detection accuracy, processing time)
% - Qualitative comparison (robustness, false positives/negatives)
% - Specific scenarios where alternative excels
% - Specific scenarios where alternative fails
% - Overall recommendation

\subsection{Summary of Session 1}

% PLACEHOLDER: Brief summary of Session 1 findings (3-4 sentences)
% - Recap the main achievements
% - Highlight the best performing approach
% - Preview transition to Session 2


% session2.tex - Session 2: Line Following
\section{Session 2: Line Following by UGV Rover}
\label{sec:session2}

% PLACEHOLDER: Brief introduction to Session 2 (2-3 sentences)
% - Explain that this section covers the line following implementation
% - Mention the goal: autonomous following of ceiling lines
% - Preview the subsections: pipeline design, tracking algorithm, calibration effects, control improvements

\subsection{Line Following Pipeline Design}
\label{subsec:following_pipeline}

% PLACEHOLDER: Describe your line following pipeline (10 points)
% This subsection should include:
% 1. Complete pipeline from image capture to robot control
% 2. Justification for each step
% 3. ROS2 architecture and node structure

\subsubsection{System Architecture}

% PLACEHOLDER: Describe the overall system architecture (4-5 sentences)
% - ROS2 nodes created (e.g., line_detector node, controller node)
% - Topic structure (image topics, velocity command topics)
% - How components communicate
% - Real-time processing considerations

\begin{figure}[H]
    \centering
    % PLACEHOLDER: Insert system architecture diagram
    \fbox{\parbox{0.8\textwidth}{\centering
        \textbf{[FIGURE: ROS2 System Architecture]}\\[1em]
        Insert a diagram showing:\\
        Camera Node $\rightarrow$ Line Detection Node $\rightarrow$ Control Node $\rightarrow$ Motor Commands
    }}
    \caption{ROS2 system architecture for line following. PLACEHOLDER: This figure should show the complete ROS2 node structure, topics, and data flow.}
    \label{fig:system_architecture}
\end{figure}

\subsubsection{Pipeline Steps}

% PLACEHOLDER: Describe each step in the line following pipeline

\paragraph{Step 1: Image Acquisition and Processing}
% PLACEHOLDER: Describe how images are continuously captured and processed
% - ROS2 camera subscriber implementation
% - Processing rate and frame skipping if needed
% - Integration with line detection algorithm from Session 1

\paragraph{Step 2: Line Feature Extraction}
% PLACEHOLDER: Describe how line features are extracted
% - Line position (x-coordinate in image)
% - Line orientation (angle)
% - Line confidence/quality metric
% - Why: these features guide the control decision

\paragraph{Step 3: Control Command Generation}
% PLACEHOLDER: Describe how control commands are computed
% - Mapping from line position to steering command
% - Speed control strategy
% - Error computation (deviation from desired position)
% - Why: translate visual feedback into motor commands

\paragraph{Step 4: Robot Actuation}
% PLACEHOLDER: Describe robot control
% - ROS2 velocity command publisher
% - Command format (linear and angular velocities)
% - Safety limits and emergency stop

\subsubsection{Implementation Details}

\begin{lstlisting}[language=Python, caption={Line following ROS2 node (simplified). PLACEHOLDER: Replace with your actual implementation.}]
import rclpy
from rclpy.node import Node
from sensor_msgs.msg import Image
from geometry_msgs.msg import Twist

class LineFollowerNode(Node):
    def __init__(self):
        super().__init__('line_follower')
        # PLACEHOLDER: Insert your initialization code

        # Subscribe to camera images
        self.image_sub = self.create_subscription(
            Image, '/camera/image_raw',
            self.image_callback, 10)

        # Publish velocity commands
        self.cmd_pub = self.create_publisher(
            Twist, '/cmd_vel', 10)

    def image_callback(self, msg):
        # PLACEHOLDER: Insert your image processing code
        # - Detect lines
        # - Compute control command
        # - Publish velocity command
        pass

# PLACEHOLDER: Add more code as needed
\end{lstlisting}

\subsection{Tracking a Specific Ceiling Line}
\label{subsec:line_tracking}

% PLACEHOLDER: Describe tracking algorithm (10 points)
% This subsection should include:
% 1. Problem description (multiple lines detected)
% 2. Solution approach
% 3. Comparison with initial version

\subsubsection{Problem Statement}

% PLACEHOLDER: Explain the multiple line problem (3-4 sentences)
% - Multiple ceiling lines detected simultaneously
% - Also ground lines, window edges, etc.
% - Robot gets confused and switches between lines
% - Need to consistently follow ONE specific line

\begin{figure}[H]
    \centering
    % PLACEHOLDER: Insert image showing multiple detected lines
    \fbox{\parbox{0.6\textwidth}{\centering [Image with Multiple Detected Lines]}}
    \caption{Problem illustration: Multiple lines detected including ceiling lines, ground lines, and window edges. PLACEHOLDER: Show an actual image with all detected lines marked.}
    \label{fig:multiple_lines}
\end{figure}

\subsubsection{Tracking Algorithm Design}

% PLACEHOLDER: Describe your tracking solution (5-6 sentences)
% Possible approaches:
% - Temporal consistency (track line closest to previous frame's line)
% - Position filtering (select line in expected region)
% - Brightness-based selection (brightest line = ceiling line)
% - Kalman filter or particle filter for tracking
% - Combination of multiple criteria

\begin{algorithm}[H]
\caption{Line Tracking Algorithm. PLACEHOLDER: Replace with your actual algorithm}
\label{alg:tracking}
\begin{algorithmic}[1]
\State \textbf{Input:} Set of detected lines $L = \{l_1, l_2, ..., l_n\}$, previous target line $l_{prev}$
\State \textbf{Output:} Current target line $l_{target}$
\State
\State \texttt{// PLACEHOLDER: Insert your tracking algorithm}
\If{first frame}
    \State $l_{target} \gets \text{SelectInitialLine}(L)$ \Comment{e.g., brightest, most central}
\Else
    \State $distances \gets \text{ComputeDistances}(L, l_{prev})$
    \State $l_{target} \gets \arg\min_{l \in L} distances[l]$ \Comment{closest to previous}
\EndIf
\State
\State \texttt{// Additional filtering based on brightness, position, etc.}
\State \Return $l_{target}$
\end{algorithmic}
\end{algorithm}

\subsubsection{Implementation Results}

% PLACEHOLDER: Show results of tracking algorithm
% - Before and after comparison
% - Demonstrate consistent tracking of single line

\begin{figure}[H]
    \centering
    \begin{subfigure}[b]{0.45\textwidth}
        \centering
        \fbox{\parbox{0.9\textwidth}{\centering [Without Tracking]}}
        \caption{Without tracking: robot switches between lines}
    \end{subfigure}
    \hfill
    \begin{subfigure}[b]{0.45\textwidth}
        \centering
        \fbox{\parbox{0.9\textwidth}{\centering [With Tracking]}}
        \caption{With tracking: consistent following}
    \end{subfigure}
    \caption{Comparison of line following behavior. PLACEHOLDER: Show actual trajectory or sequence of frames demonstrating improved tracking.}
    \label{fig:tracking_comparison}
\end{figure}

\subsubsection{Analysis}

% PLACEHOLDER: Analyze why tracking works better (4-5 sentences)
% - Quantitative improvement if possible (success rate, trajectory smoothness)
% - Qualitative improvements observed
% - Remaining challenges
% - Parameter tuning discussion

\subsection{Impact of Camera Calibration}
\label{subsec:calibration}

% PLACEHOLDER: Discuss camera calibration effects (10 points)
% This subsection should include:
% 1. Comparison between raw and rectified images
% 2. Impact on line detection and following performance
% 3. Consideration of whether additional calibration is needed

\subsubsection{Camera Calibration Background}

% PLACEHOLDER: Brief explanation of camera calibration (3-4 sentences)
% - What is camera calibration (correcting lens distortion)
% - Why it matters (straight lines should appear straight)
% - Calibration parameters used

\subsubsection{Experimental Comparison}

% PLACEHOLDER: Describe experiments comparing raw vs rectified images
% - How experiments were conducted (same trajectory with both image types)
% - Metrics used for comparison

\begin{figure}[H]
    \centering
    \begin{subfigure}[b]{0.45\textwidth}
        \centering
        \fbox{\parbox{0.9\textwidth}{\centering [Raw Image]}}
        \caption{Raw camera image}
    \end{subfigure}
    \hfill
    \begin{subfigure}[b]{0.45\textwidth}
        \centering
        \fbox{\parbox{0.9\textwidth}{\centering [Rectified Image]}}
        \caption{Rectified image}
    \end{subfigure}
    \caption{Comparison of raw and rectified images. PLACEHOLDER: Show actual images highlighting distortion correction, especially at image edges.}
    \label{fig:calibration_images}
\end{figure}

\subsubsection{Performance Analysis}

% PLACEHOLDER: Analyze the impact on performance (5-6 sentences)
% - Did rectification improve line detection accuracy?
% - Impact on line following stability
% - Are there remaining distortions?
% - Is the current calibration sufficient or is additional calibration needed?
% - Specific scenarios where calibration matters most (e.g., lines near image edges)

\begin{table}[H]
\centering
\caption{Performance comparison: raw vs. rectified images. PLACEHOLDER: Replace with actual measurements.}
\label{tab:calibration}
\begin{tabular}{@{}lcc@{}}
\toprule
\textbf{Metric} & \textbf{Raw Images} & \textbf{Rectified Images} \\ \midrule
Line detection accuracy (\%) & XX & XX \\
Tracking stability (rating 1-10) & X & X \\
Control smoothness (std dev of angular velocity, rad/s) & X.XX & X.XX \\
False positive rate (\%) & XX & XX \\ \bottomrule
\end{tabular}
\end{table}

\subsection{Control Improvement and Oscillation Reduction}
\label{subsec:control_improvement}

% PLACEHOLDER: Describe control improvements (10 points)
% This subsection addresses the "open question"
% Should include:
% 1. Problem identification (oscillation, swinging behavior)
% 2. Solution approach (e.g., PID controller)
% 3. Results showing improvement

\subsubsection{Problem: Oscillatory Behavior}

% PLACEHOLDER: Describe the oscillation problem (4-5 sentences)
% - Robot swings left and right while following line
% - Causes of oscillation (pure proportional control, delays, etc.)
% - Why this is problematic (inefficient, potential instability)
% - Need for more sophisticated control

\begin{figure}[H]
    \centering
    % PLACEHOLDER: Insert diagram or plot showing oscillatory trajectory
    \fbox{\parbox{0.7\textwidth}{\centering [Oscillatory Trajectory Plot]}}
    \caption{Oscillatory behavior without control improvement. PLACEHOLDER: Show actual trajectory or time-series plot of lateral deviation.}
    \label{fig:oscillation}
\end{figure}

\subsubsection{Control Algorithm Design}

% PLACEHOLDER: Describe your control solution (5-7 sentences)
% Common approaches:
% - PID controller (Proportional-Integral-Derivative)
% - Damped proportional control
% - Model predictive control
% - Fuzzy logic controller
% Explain the choice and design rationale

\paragraph{PID Controller Design}
% PLACEHOLDER: If using PID, describe each component
% - Proportional term: responds to current error
% - Integral term: eliminates steady-state error
% - Derivative term: reduces oscillation and overshoot
% - Parameter tuning process

The control law is given by:
\begin{equation}
    u(t) = K_p \cdot e(t) + K_i \cdot \int_0^t e(\tau) d\tau + K_d \cdot \frac{de(t)}{dt}
\end{equation}

where:
\begin{itemize}
    \item $u(t)$ is the control output (angular velocity command)
    \item $e(t)$ is the error (deviation of line from image center)
    \item $K_p, K_i, K_d$ are the PID gains
\end{itemize}

% PLACEHOLDER: Describe your specific parameter values and tuning process
The PID parameters were tuned as follows:
\begin{itemize}
    \item $K_p = $ \texttt{[PLACEHOLDER: your value]} (chosen because...)
    \item $K_i = $ \texttt{[PLACEHOLDER: your value]} (chosen because...)
    \item $K_d = $ \texttt{[PLACEHOLDER: your value]} (chosen because...)
\end{itemize}

\subsubsection{Implementation}

\begin{lstlisting}[language=Python, caption={PID controller implementation. PLACEHOLDER: Replace with your actual code.}]
class PIDController:
    def __init__(self, Kp, Ki, Kd):
        self.Kp = Kp
        self.Ki = Ki
        self.Kd = Kd
        self.prev_error = 0
        self.integral = 0

    def compute(self, error, dt):
        # PLACEHOLDER: Insert your PID implementation
        self.integral += error * dt
        derivative = (error - self.prev_error) / dt
        output = self.Kp * error + self.Ki * self.integral + self.Kd * derivative
        self.prev_error = error
        return output
\end{lstlisting}

\subsubsection{Results and Comparison}

% PLACEHOLDER: Show results comparing before and after control improvement

\begin{figure}[H]
    \centering
    \begin{subfigure}[b]{0.45\textwidth}
        \centering
        \fbox{\parbox{0.9\textwidth}{\centering [Trajectory Before]}}
        \caption{Before: simple proportional control}
    \end{subfigure}
    \hfill
    \begin{subfigure}[b]{0.45\textwidth}
        \centering
        \fbox{\parbox{0.9\textwidth}{\centering [Trajectory After]}}
        \caption{After: PID control}
    \end{subfigure}
    \caption{Trajectory comparison. PLACEHOLDER: Show actual trajectories or error plots demonstrating reduced oscillation.}
    \label{fig:control_comparison}
\end{figure}

\begin{table}[H]
\centering
\caption{Control performance comparison. PLACEHOLDER: Replace with actual measurements.}
\label{tab:control}
\begin{tabular}{@{}lcc@{}}
\toprule
\textbf{Metric} & \textbf{Before} & \textbf{After} \\ \midrule
Oscillation amplitude (pixels) & XX & XX \\
Settling time (seconds) & X.X & X.X \\
Overshoot (\%) & XX & XX \\
Tracking error RMS (pixels) & XX & XX \\
Completion time (seconds) & XX & XX \\ \bottomrule
\end{tabular}
\end{table}

\subsubsection{Analysis}

% PLACEHOLDER: Analyze why the improved control works better (5-6 sentences)
% - Quantitative improvements from table
% - Qualitative improvements observed
% - Explanation of how control components contribute
% - Trade-offs (e.g., slower response vs. stability)
% - Remaining limitations

\subsection{Bonus Question: Camera Selection}
\label{subsec:camera_selection}

% PLACEHOLDER: Address the bonus question (10 bonus points)
% This subsection should include:
% 1. Comparison of using OAK-D camera vs. current camera
% 2. Considerations for ceiling line vs. ground line following
% 3. Optimal tilt angles for both tasks

\subsubsection{OAK-D Camera Characteristics}

% PLACEHOLDER: Describe the OAK-D camera (3-4 sentences)
% - Stereo depth capability
% - Higher resolution
% - Built-in AI processing
% - Field of view differences

\subsubsection{Ceiling Line Following with OAK-D}

% PLACEHOLDER: Discuss using OAK-D for ceiling line following (4-5 sentences)
% - Potential advantages (depth information, better resolution)
% - Potential disadvantages (different field of view, mounting considerations)
% - Whether depth information would be useful for this task
% - Overall suitability

\subsubsection{Ground Line Following Comparison}

% PLACEHOLDER: Compare ceiling vs. ground line following (5-6 sentences)
% - Key differences in the task
% - Which camera is better suited for ground line following and why
% - Importance of depth perception for ground navigation
% - Obstacle avoidance considerations

\subsubsection{Optimal Camera Tilt Angles}

% PLACEHOLDER: Discuss optimal tilt angles (5-6 sentences)
% - Current tilt angle for ceiling line following
% - Recommended tilt angle for ground line following
% - Trade-offs in tilt angle selection (field of view vs. line visibility)
% - How tilt affects line detection difficulty

\begin{figure}[H]
    \centering
    % PLACEHOLDER: Insert diagram showing camera angles
    \fbox{\parbox{0.7\textwidth}{\centering
        \textbf{[DIAGRAM: Camera Tilt Angles]}\\[1em]
        Show side view of rover with camera at different tilt angles\\
        for ceiling vs. ground line following
    }}
    \caption{Optimal camera tilt angles for different tasks. PLACEHOLDER: Illustrate recommended angles with justification.}
    \label{fig:camera_angles}
\end{figure}

\subsection{Summary of Session 2}

% PLACEHOLDER: Brief summary of Session 2 findings (3-4 sentences)
% - Recap the main achievements (complete line following system)
% - Highlight key improvements (tracking, control)
% - Summarize overall success and challenges


% conclusion.tex - Conclusion section
\section{Conclusion}
\label{sec:conclusion}

% PLACEHOLDER: Opening paragraph (2-3 sentences)
% - Restate the main objective of the assignment
% - Briefly mention that the objectives were achieved
% - Preview the structure of the conclusion

\subsection{Summary of Achievements}
\label{subsec:achievements}

% PLACEHOLDER: Summarize main achievements (5-7 sentences)
% This should cover both sessions:
% Session 1:
% - Successfully designed and implemented line detection pipeline
% - Evaluated multiple edge detection algorithms
% - Identified strengths and limitations of each approach
% Session 2:
% - Implemented complete autonomous line following system
% - Developed robust line tracking algorithm
% - Improved control stability through [PID/other] controller
% - Achieved successful autonomous navigation

This assignment successfully achieved the following:

\paragraph{Line Detection (Session 1)}
% PLACEHOLDER: List specific Session 1 achievements
\begin{itemize}
    \item Designed a complete line detection pipeline from raw camera images to detected ceiling lines
    \item Implemented and evaluated Canny edge detector with [success rate/accuracy]
    \item Tested the algorithm on a complete rosbag recording, identifying success and failure cases
    \item Developed an alternative detection algorithm [name] that [achieved X\% improvement / performed differently]
\end{itemize}

\paragraph{Line Following (Session 2)}
% PLACEHOLDER: List specific Session 2 achievements
\begin{itemize}
    \item Implemented a complete ROS2-based line following system
    \item Developed a tracking algorithm to consistently follow a single ceiling line despite multiple distractors
    \item Evaluated the impact of camera calibration, finding that [rectified/raw] images provided better performance
    \item Designed and implemented a [PID/other] controller that reduced oscillation by [X\%] and improved tracking stability
    \item Successfully demonstrated autonomous line following on the UGV Rover
\end{itemize}

\subsection{Key Findings}
\label{subsec:findings}

% PLACEHOLDER: Discuss key findings and insights (6-8 sentences)
% This should include:
% - Which line detection algorithm worked best and why
% - Most challenging aspects of the task
% - Importance of tracking vs. detection
% - Impact of control algorithm on overall performance
% - Unexpected discoveries or challenges
% - Lessons learned about vision-based robot control

Several important findings emerged from this work:

\paragraph{Line Detection Performance}
% PLACEHOLDER: Summarize findings about line detection (3-4 sentences)
% - Which algorithm performed best overall
% - Key factors affecting performance (lighting, clutter, line characteristics)
% - Importance of preprocessing and filtering

\paragraph{Tracking and Control}
% PLACEHOLDER: Summarize findings about tracking and control (3-4 sentences)
% - Tracking is as important as detection
% - Control algorithm design significantly affects stability
% - Trade-offs between responsiveness and stability

\paragraph{Camera Calibration}
% PLACEHOLDER: Summarize findings about calibration (2-3 sentences)
% - Impact of lens distortion correction
% - Whether additional calibration would help

\subsection{Limitations and Challenges}
\label{subsec:limitations}

% PLACEHOLDER: Discuss limitations and challenges (5-7 sentences)
% Be honest about what didn't work perfectly or what challenges remain:
% - Environmental dependencies (lighting, line visibility)
% - Computational performance limitations
% - Robustness issues (what situations cause failures)
% - Hardware limitations
% - Algorithm limitations

Despite the successful implementation, several limitations were encountered:

\begin{itemize}
    \item \textbf{Lighting Sensitivity:} % PLACEHOLDER: Describe how changing lighting affects performance
    \item \textbf{Computational Load:} % PLACEHOLDER: Discuss processing time and real-time constraints
    \item \textbf{Robustness:} % PLACEHOLDER: Discuss situations where the system fails or struggles
    \item \textbf{Generalization:} % PLACEHOLDER: Discuss how well the system would work in different environments
    \item \textbf{Safety Considerations:} % PLACEHOLDER: Discuss challenges with avoiding obstacles, detecting end of path
\end{itemize}

\subsection{Future Work}
\label{subsec:future}

% PLACEHOLDER: Suggest future improvements and extensions (5-7 sentences)
% This should be forward-looking and realistic:
% - Algorithmic improvements (machine learning, better filtering)
% - System improvements (faster processing, better hardware)
% - Extended functionality (obstacle avoidance, multi-line paths)
% - More robust solutions (adaptive algorithms, learning-based approaches)

Several directions for future work could improve and extend this system:

\paragraph{Algorithm Improvements}
% PLACEHOLDER: Suggest algorithmic improvements (2-3 items)
\begin{itemize}
    \item \textbf{Machine Learning Approaches:} Train a neural network for line detection that is robust to varying lighting and environmental conditions
    \item \textbf{Adaptive Control:} Implement adaptive PID gains that adjust based on velocity and line characteristics
    \item \textbf{Predictive Tracking:} Use Kalman filtering or particle filtering for more robust line tracking
\end{itemize}

\paragraph{System Enhancements}
% PLACEHOLDER: Suggest system improvements (2-3 items)
\begin{itemize}
    \item \textbf{Sensor Fusion:} Combine camera with other sensors (IMU, odometry) for improved stability
    \item \textbf{End-of-Line Detection:} Implement robust detection of line endpoints to safely stop the rover
    \item \textbf{Obstacle Avoidance:} Integrate obstacle detection and avoidance capabilities
\end{itemize}

\paragraph{Extended Applications}
% PLACEHOLDER: Suggest extended applications (2-3 items)
\begin{itemize}
    \item \textbf{Ground Line Following:} Adapt the system for traditional ground-based line following
    \item \textbf{Complex Paths:} Handle intersections, branching paths, and path selection
    \item \textbf{Multi-Robot Coordination:} Extend to multiple rovers following the same line
\end{itemize}

\subsection{Concluding Remarks}
\label{subsec:concluding}

% PLACEHOLDER: Final concluding paragraph (3-5 sentences)
% - Reaffirm the success of the assignment
% - Emphasize the main contribution or achievement
% - Broader context: importance of vision-based control for robotics
% - Final thought about what was learned

% Example structure:
% "This assignment successfully demonstrated the design and implementation of a complete vision-based
% line following system for a UGV Rover. Through systematic development of line detection algorithms,
% tracking strategies, and control systems, we achieved autonomous navigation following ceiling-mounted
% lines. The insights gained regarding the interplay between perception, tracking, and control are
% fundamental to mobile robotics and autonomous systems. This work highlights both the capabilities
% and challenges of vision-based robot control, providing a foundation for more sophisticated
% autonomous navigation systems."


% ==================== REFERENCES ====================
\bibliographystyle{plain}
\bibliography{references}
% Alternatively, you can manually add references:
\begin{thebibliography}{9}
\bibitem{dudek2024}
G. Dudek and M. Jenkin,
\textit{Computational Principles of Mobile Robotics}, 2nd ed.
Cambridge University Press, 2024.

\bibitem{opencv}
OpenCV Documentation,
\textit{Edge Detection using OpenCV},
\url{https://docs.opencv.org/4.x/da/d22/tutorial_py_canny.html},
Accessed: \today.

% PLACEHOLDER: Add more references as needed for:
% - ROS2 documentation
% - Line following GitHub repositories referenced
% - Any additional academic papers or resources used
\end{thebibliography}

\end{document}
