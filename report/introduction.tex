% introduction.tex - Introduction section
\section{Introduction}
\label{sec:introduction}

% PLACEHOLDER: Opening paragraph (2-3 sentences)
% - Introduce the general context: autonomous robots and their need for visual feedback
% - Mention the importance of line following in robotics applications
% - Example: "Autonomous mobile robots rely heavily on visual feedback to navigate their environment.
%   Line following is a fundamental task in robotics, with applications ranging from warehouse automation
%   to autonomous vehicles maintaining lane positions."

\subsection{Motivation and Background}
\label{subsec:motivation}

% PLACEHOLDER: Motivation paragraph (3-4 sentences)
% - Explain why line detection and following is important
% - Discuss real-world applications (autonomous vehicles, warehouse robots, etc.)
% - Mention the specific context of this lab (UGV Rover following ceiling lines)
% - Example: "In industrial environments, autonomous guided vehicles (AGVs) use line following to
%   transport materials efficiently. This assignment explores line following in a unique scenario:
%   following ceiling-mounted light lines rather than ground markers."

\subsection{Assignment Objectives}
\label{subsec:objectives}

% PLACEHOLDER: List the main objectives of the assignment
% This should be a clear enumeration of what the lab aims to achieve:
The main objectives of this assignment are:

\begin{enumerate}
    \item \textbf{Line Detection (Session 1):}
    % PLACEHOLDER: Describe the line detection objectives
    % - Design and implement a line detection pipeline
    % - Evaluate Canny edge detector performance
    % - Test on provided rosbag data
    % - Explore and compare alternative line detection algorithms

    \item \textbf{Line Following (Session 2):}
    % PLACEHOLDER: Describe the line following objectives
    % - Implement a complete line following system
    % - Develop tracking algorithm to follow specific ceiling line
    % - Evaluate impact of camera calibration
    % - Address control challenges (e.g., oscillation, stability)

    \item \textbf{System Integration:}
    % PLACEHOLDER: Describe integration objectives
    % - Integrate computer vision algorithms with ROS2
    % - Implement real-time control based on visual feedback
    % - Test and validate on physical UGV Rover platform
\end{enumerate}

\subsection{Technical Approach}
\label{subsec:approach}

% PLACEHOLDER: Brief overview of technical approach (3-5 sentences)
% - Mention the UGV Rover platform and its capabilities
% - Describe the ROS2 framework used
% - Outline the general pipeline: image capture → processing → line detection → control → actuation
% - Mention key tools: OpenCV for image processing, Python for implementation
% - Example: "This assignment uses a UGV Rover equipped with a pan-tilt camera to capture images of
%   ceiling lines. The Robot Operating System 2 (ROS2) framework provides the infrastructure for
%   real-time image processing and robot control."

\subsection{Report Organization}
\label{subsec:organization}

% PLACEHOLDER: Outline the structure of the report
% This should match the actual structure of your document
The remainder of this report is organized as follows:
\begin{itemize}
    \item Section \ref{sec:session1} describes the line detection pipeline design and implementation, including comparison of different edge detection algorithms.
    \item Section \ref{sec:session2} presents the line following system, including tracking algorithms, camera calibration effects, and control improvements.
    \item Section \ref{sec:conclusion} summarizes the key findings, discusses limitations, and suggests future improvements.
\end{itemize}
